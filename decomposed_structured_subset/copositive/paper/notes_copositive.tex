\documentclass{article}
\usepackage[utf8]{inputenc}
\usepackage[margin=1in]{geometry}

\usepackage{amsmath} % assumes amsmath package installed
\usepackage{amssymb}  % assumes amsmath package installed
\usepackage{amsthm}
\usepackage{bm}
\usepackage{mathtools}

% \newtheorem{mythm
% \newtheorem{mydef}{Definition}
% %will define the mydef environment; if you use it like this:

% \begin{mydef}
% Here is a new definition
% \end{mydef}


\newcommand{\tr}{{\mathsf T}}
\DeclarePairedDelimiter{\abs}{\lvert}{\rvert}
\DeclarePairedDelimiter{\norm}{\lVert}{\rVert}
\DeclarePairedDelimiter{\ceil}{\lceil}{\rceil}
\DeclarePairedDelimiter{\Tr}{\text{Tr}(}{)}
\DeclarePairedDelimiter{\rank}{\text{rank}(}{)}
\DeclarePairedDelimiter{\diag}{\text{diag}(}{)}
\DeclarePairedDelimiterX{\inp}[2]{\langle}{\rangle}{#1, #2}
\newcommand{\R}{\mathbb{R}} 

\newcommand{\Z}{\mathbb{Z}}
\newcommand{\F}{\mathcal{F}}
\newcommand{\gs}{\mathcal{G}}
\newcommand{\es}{\mathcal{E}}
\newcommand{\vs}{\mathcal{V}}
 \newcommand{\psd}{\mathbb{S}}
\newcommand{\cs}{\mathcal{C}}
\newcommand{\ks}{\mathcal{K}}
\newcommand{\fw}{\mathcal{F}\mathcal{W}}
\newcommand{\us}{\mathcal{U}}
\newcommand{\psds}{\mathcal{S}}
\newcommand{\dd}{\text{\emph{DD}}}
\newcommand{\sdd}{\text{\emph{SDD}}}
\newcommand{\N}{\mathcal{N}}
\newcommand{\dnn}{\mathcal{D}\mathcal{N}\mathcal{N}}
\newcommand{\cdd}{\cs\dd}
\newcommand{\csdd}{\cs\sdd}
\newcommand{\cp}{\mathcal{C}\mathcal{P}}
\newcommand{\cop}{\mathcal{C}\mathcal{O}\mathcal{P}}
\newtheorem{theorem}{Theorem}
\newtheorem{remark}{Remark}
\newtheorem{proposition}{Proposition}


\title{Copositive Approximations Decomposed Structured Subsets}
\author{Jared Miller}
\date{\today}

\begin{document}

\maketitle

\section{Introduction}

\maketitle


Copositive and Completely Positive Programming are NP-hard convex conic optimization problems. Some standard cones in optimization include:

\begin{align*}
    \psd^n &= \{M \in \R^{n \times n} \mid M = M^{\tr}\} \\
    \psd_+^n &= \{M \in \psd^n \mid x^{\tr} M x \geq 0 \ \forall x \in \R\} \\
    \N_+^n &= \{M \in \psd^N \mid M_{ij} \geq 0\}
\end{align*}

These are the set of symmetric matrices, positive semidefinite symmetric matrices, and nonnegative matrices. The Copositive cone $\cop^n$ and Completely Positive cone $\cp^n$ are defined as:

\begin{align*}
    \cop^n &= \{M \in \psd^n \mid x^{\tr} M x \geq 0 \ \forall x \in \R^n_+\} \\
    \cp^n &= \{M \in \psd^n \mid M = U U^{\tr}, U \geq 0 \}
\end{align*}

Copositive matrices only necessitate nonnegative quadratic forms over the real orthant, and completely positive matrices have nonnegative factorizations (more general than PSD matrices which require  $\exists U \mid M = U U^{\tr}$). By these properties, $\cp^n \subset \psd_+^n \subset \cop^n$ \cite{berman2003completely}. The width of $U$ necessary to form this completely positive decomposition is known as the completely positive rank $\text{cp-rank}(M)$. These two cones are dual to each other: $\cop^n = (\cp^n)^*$.

The stability number of a graph is the optimum value of a copositive program. Let $G$ be a graph with adjacency matrix $A_G$, $I$ be the identity matrix and $J = 1 1^{\tr}$ denote the all-ones matrix. The stability number $\alpha(G)$ is:

\[\alpha_G = \max_\lambda \lambda \mid \lambda(A_G + I) - J \in \cop^n\]

See \cite{de2002approximation} on how to derive this copositive program, and \cite{bomze2012copositive} for a survey of copositive optimization.

\section{Inner and Outer Approximations}

Inner and outer approximations of these cones can be developed through semidefinite programming. As a completely positive matrix is both PSD and has nonnegative entries, the Doubly Nonnegative cone $\dnn^n= \N^n \cap \psd_+^n \supset \cp^n$. By duality, $(\dnn^n)^* = \N^n + \psd_+^n \subset \cop^n$. Parrilo defined a hierarchy of LMI-representable cones that successively approximate the copositive cone \cite{parrilo2000structured}. Given a matrix $M$ with quadratic form $q_M(x) = x^T M x$, the $r$-th approximation to the copositive cone is:

$\mathcal{K}^r = \{M \mid \sum_i^n(x_i^2) q_M(x^{\circ^2}) \in SOS\}$

$x^{\circ^2} = x \circ x$ is the elementwise squaring of the entries of $x$, and $SOS$ is the set of polynomials that can be represented as the sum of squared elements. The dual cones $(\mathcal{K}^r)^*$ will then produce successively smaller outer approximations to the completely positive cone. See \cite{de2002approximation, pena2007computing} for additional copositive hierarchies.

% de Klerk and Pasechnik use a Polya type result

Inner approximations of the completely positive cone can be found by exploiting properties of the $\dnn$ cone. For $k = 1 \ldots 4$, $\dnn^k = \cp^k$. When $k \geq 5$, $\dnn^k \supset \cp^k$. A study of matrices in $\dnn^5$ that are not in $\cp^5$ is conducted in \cite{burer2009difference}. This inclusion allows for scaled diagonally dominant nonnegative approximations \cite{gouveia2018inner}, as well as extensions to factor width $\fw^n_{k \leq 4}$ \cite{ding2018higherorder}. Block Factor width  2 matrices with $2\times2$ blocks can also be used \cite{zheng2019block} (I have done this successfully, and it has better performance in time and cost as compared to SDD).

\section{Employing Structure}

Given the following conic program with cone $K$:

\begin{align}%[t]
    p^* =\min_{X} \quad & \inp{C}{X} \nonumber \\
    & \inp{A_i}{X} = b_i, i = 1, \ldots, m,\label{Eq:SDPprimal}\\
     & X \in K,   \nonumber \\
\end{align}

Let $\es$ be the aggregate sparsity pattern of the matrices $C, A_i$ with maximal cliques $\{\cs_k\}_{k=1}^p$. In our previous paper, we defined the cone $K(\es, ?)$ as the decomposed structured subset where each clique is inside $K$. By Grone's theorem, the set of matrices with positive semidefinite completion is equal to $\psd_+(\es, ?)$ when $\es$ is chordal \cite{grone1984positive}. Partially copositive matrices can be completed to copositive matrices if all diagonal elements are specified, and this will hold during optimization over the graph augmented with self loops $\es^*$ \cite{hogben2005copositive}. Completely positive completions require a stronger block-clique completion \cite{berman2003completely}. A graph is block-clique if all maximal cliques intersect in at most one vertex. Block-clique graphs are a superset of chordal graphs.

If the copositive program has is sparse, it can be chordal extended and then decomposed through this copositive completion result. Copositive hierarchies such as $\dnn^*$ and $\mathcal{K}^*$ will be more accurate in finding inner approximations after decomposition. Completely positive optimization requires block-clique-extension, and then will produce more accurate inner completely positive approximations. By duality, this method allows for outer approximations over $\cp^n(\es, 0)$ when $\es$ is chordal and inner approximations of $\cop^n(\es, 0)$ when $\es$ is block-clique.

% A proof of Agler's theorem for $\psd_+^n(\es, 0)$ is available in \cite{Kakimura2010}, we adapted this for use in DD and SDD matrices. I believe all steps here can be adapted for use in Copositive matrices (will need to check schur complement), which gives us a method to 

Symmetry reduction allows for another class of possible decompositions. In the case of stability number computations, the symmetry group in question is the automorphism group of the graph. Symmetry reduction is possible in copositive programming \cite{dobre2015exploiting}, I will need to look if this is possible for completely positive programming as well.


\section{Next Steps}
Find numerical examples to show improvement. Profile different structured subsets. Clean this up and turn it into a paper.

\begin{theorem} [Grone \cite{grone1984positive}] \label{T:GroneTheorem}
     Let $\mathcal{G}(\mathcal{V},\mathcal{E})$ be a chordal graph with a set of maximal cliques $\{\mathcal{C}_1,\mathcal{C}_2, \ldots, \mathcal{C}_p\}$. Then, $X\in\mathbb{S}^n_+(\mathcal{E},?)$ if and only if
     $$ X_k = E_{\mathcal{C}_k} X E_{\mathcal{C}_k}^\tr \in \mathbb{S}^{\vert \mathcal{C}_k \vert}_+,
    \qquad k=1,\,\ldots,\,p.$$
    \end{theorem}

\medskip
\bibliographystyle{IEEEtran}
\bibliography{IEEEabrv,bib_copositive.bib}

\end{document}
